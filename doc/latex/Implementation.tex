For the implementation of the model we used MATLAB. In order to keep the code readable as the project grows, we split the main functionalities into different files.
\begin{figure}[H]
	\centering
	\includegraphics[scale=0.5]{Dependencies.pdf}
	\caption{Dependencies of the different MATLAB scripts. The arrows indicate calls of functions}
\end{figure}
The only callable file is \textit{init.m}. It defines all global variables like the mean lifetime of the pheromones, the number of ants and the parameters of the simulation. All the objects, which are important to the simulation are also created in this file. This is done by a call of \textit{gen\_sources}, \textit{gen\_colonies}, \textit{gen\_graph} and \textit{gen\_ants}. The generated structures are then fed into the \textit{simulation} script, where a simulation loop is called and the position of the ants is updated by \textit{ant\_move} for the specified number of iterations. The relevant data is then stored and processed by the \textit{analyze} function. In the end the result of the analysis is visualized by the \textit{draw} function.
\subsection{Graph}
The network, on which the ants move is given by a graph. This graph consists of two different types of nodes. At first there is the traffic node. Traffic nodes are introduced, where three or more roads come together and the ant is forced to choose one of those paths. The other type of node is the source node. Each source node is the start of a colony. However the members of a colony do not see their own starting node as a food source (they can not get their food from their starting node). But the starting nodes of other colonies are seen as food sources. Because we have to be able to control the structure of the graph, it can be read from a .txt file containing the information of all nodes and edges. Those text files are randomly generated by a python script with the method of power law distribution.
\subsection{Ant behaviour}
The movement and decision of the ants is based on the model described in chapter \ref{modeldesc}. The script responsible for the update of the position at every time step is called \textit{ant\_move}. There the next position of the ant on the graph is determined. The ant has a variable to store the progress on the edge. As soon, as this progress reaches the weight of the edge a node has been reached. If the node is a food source, different from its starting node the ant will bring some of the food back to the colony. By doing so the food source will experience a devaluation. The ant follows exactly the same path back to its nest, that it took on the way to the source. On the way back it will deposit pheromones, of quality depending on the quality of the found food source. The pheromones are saved in a variable of the edges.
If the found node is a traffic node, then the ant has to make a decision between the n adjacent paths. This decision happens in the \textit{ant\_decision} function. As seen in formula \ref{multiDecisions} the decision depends on the concentration of pheromones on the path. This concentration is calculated by
\begin{equation}
c = \frac{pheromones}{weight}
\end{equation}
Where weight is the length of an edge. The ant is free to choose from all the edges on the node, except for the one it used to get there.
However this freedom presents another problematic. Because the ant is free to choose between all the edges it is allowed to walk in cycles. Those cycles would then be reinforced by the trail laying on the way back. To solve this we still allow the ants to walk in circles, but when they do the circle will be removed from the path vector, which is used to find the way back to the nest.
The U-turns are implemented exactly as described in equation \ref{UTurn}. At every time step this probability is calculated and if a random number between 0 and 1 is lower then the calculated probability, then a U-turn happens and the ant walks back to the last visited node. 