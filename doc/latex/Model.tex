The model is based on research on the ant Lasius niger, which was published in Camazin chapter 13 in 2001. This paper describes the process of trail formation in ants. This phenomenon can easily observed in the wild, by placing some sugar solution. After a short while the food will be discovered and shortly after the number of ants at the food source will increase rapidly, until eventually it stabilizes and a well established trail of ants can be observed from the nest to the sugar.  To study this behaviour, several experiments have been conducted on a colony of Lasius niger. However the research was done under Laboratory conditions, so that the terrain could be controlled and therefore the experiments could be repeated. The terrain was constructed in such a way, that there was one path leading away from the nest, which then split into two paths with a food source at the end of both paths. The length, as well as the quality of the food sources could be controlled. Then the behaviour of the ants was observed and formulated as a mathematical model.
\subsection{Trail laying}
It was found, that the ants would place chemicals, so called pheromones, which brings other ants to follow the laid path. When an ant finds a food source it will lay pheromones on the way back to the nest and on the subsequent trips to the source. This acts as a positive feedback for the other ants, which are likely to follow that trail. The behaviour of trail laying is only observed by ants, who found food. The intensity of trail laying is dependant on the quality of the found food. The better the food source, the more pheromones will be deposited. The frequency of laying pheromones is also reduced based on the current direction of the ant. If the ant is walking in a trajectory with a bad angle to the direct route to the nest less pheromones will be placed. The placed chemicals also evaporate over time. They undergo an exponential decay given by the formula
$$\frac{dC}{dt}=-\frac{1}{l}C$$
where l is the lifetime of the pheromones and C the concentration.
\subsection{Binary choices}
In the experiment there was a choice the ants had to make, when the path split into two and they have to decide, whether to go left or right. This decision is made based on the concentration of pheromones on each branch. SO that the branch with a higher concentration of pheromones on it has a better chance of being chosen. For that decision a formula for the probability of choosing one of the two branches was found.
$$ P_L = \frac{(k+C_L)^n}{(k+C_L)^n+(k+C_R)^n} \ and \  P_R = 1-P_L$$
n is the non-linearity factor. with a high value of n a branch with little more pheromones then the other will be strongly favoured. With experiments it is approximately found that $n=2$. k stands for the likelihood of the ant taking a path with no pheromones on it. P is the probability of choosing the left or right path and C the concentration of pheromones on each path. 
\subsection{U-Turns}
It is often observed, that ants on a trail make an U-turn, return to the common branch point and follow the other path. After a U-turn the ant does not engage in trail laying, until it has gone back to the branch and follows the new path. There are two causes of this behaviour. If the amount of pheromones on the trail is low the ant may turn around. The second reason is direction based. If the orientation to the target is not good the probability of an U-turn is high. This probability can be modelled by the equation
$$P=\frac{P_0}{1+\alpha C}$$
$P_0$ stands for the probability of an direction based U-turn. $\alpha$ describes the importance of the pheromone based U-turn
\subsection{Extensions}
For our project we could not use the model as given in the paper. For that reason the model was changed in some aspects to ease the implementation and add additional functionality. Because the goal of this paper is to model more complicated networks then the one with only one binary decision, formula ??????? has to be adapted to support a choice between more than just two paths. 
$$P_i = \frac{(k+C_i)^n}{\sum{_{j=1}^{nr}(k+C_j)^n}}$$