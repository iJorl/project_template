The model is based on research on the ant Lasius niger, which was published in Camazin chapter 13 in 2001. This paper describes the process of trail formation in ants. This phenomenon can easily observed in the wild, by placing some sugar solution. After a short while the food will be discovered and shortly after the number of ants at the food source will increase rapidly, until eventually it stabilizes and a well established trail of ants can be observed from the nest to the sugar. To study this behaviour, several experiments have been conducted on a colony of Lasius niger. However the research was done under Laboratory conditions, so that the terrain could be controlled and therefore the experiments could be repeated. The terrain was constructed in such a way, that there was one path leading away from the nest, which then split into two paths with a food source at the end of both paths. The length, as well as the quality of the food sources could be controlled. Then the behaviour of the ants was observed and formulated as a mathematical model.
