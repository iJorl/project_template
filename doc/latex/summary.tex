The main goal of this project was to simulate the behaviour of multiple ant colonies interacting in a network and then measure the productivity. Furthermore once we wanted to apply two strategies to optimize the productivity in the network by reallocating ants. The model was applied to generated graphs to produce a more network-independent conclusion and to the real world graph of the Swiss Railroad network.

Simulating the behaviour of multiple colonies leads to a stable equilibrium for the global production. Applying the two different strategies to increase the result also leads to a stable global production, but it only differs slightly from the base result. When analysing the results over multiple simulations the difference is quite small and no significant improvement can be found when applying the redistribution strategies.

When the model is applied to the Swiss Rail Graph it yields similar results for all three ant population strategies regarding the global productivity of the network. The ants seem to prefer to exploit sources that are not shared with other colonies. This requires implicit collaboration between the ants and seems to be the desired state among all three strategies and therefore they only differ slightly. When we focus on the preferred edges of the ant colonies, they resemble the routes with high traffic volume in the real rail network.

There are some points to continue with this project. As seen in the simulation results, the ants are unable to make use of the redistribution and exploit sources with higher quality (higher regeneration rate). It would be interesting to see if one could make use of the redistribution by using another quality measure to further differentiate the sources from each other. This could be done by setting an individual food limit or adding a separate quality measurement. Another aspect than one could further investigate is to add sources that are no colonies them self. This could resemble destinations that are desirable to reach, but should not get additional resources (as trains).