\subsection{Generated graphs} \label{results1}
To answer the first part of the research question, which is to compare the effects of no, global and local reallocation.For the simulation we chose to simulate for 1000 time steps and use 30 simulations for each strategy. The regeneration rate of the sources was chosen randomly with the best rate being 10 times higher then the lowest. This is about in the range, that we found the amount of passengers at a train station per day to be. The number of colonies was chosen randomly between two and ten. Each of those colonies has 200 ants. We generated 40 different graphs and run our model on all of them. The simulation on many different graphs, with different regeneration rates and different amount of colonies should give us the ability to make a statistical statement about our results.

The results of the simulations for each of the graphs looks very similar. The total productivity, which we wanted to optimize is very similar for all of the three strategies, and with the error margin being quite high it is not possible to draw the conclusion, that one of the strategies is better then the other ones. 
\subsubsection{Interpretation}
By looking at the amount of pheromones on the edges we assume the scenario that every colony exploits a source, which is not used by the other colonies. This sign of cooperation shows, that the sources run low quickly and therefore the competition for a source is not worth. Because all colonies use this strategy and the sources are emptied quickly. Hence the reallocation of ants to a new colony does not have a great impact on the productivity, since the sources can not be exploited any more. 

We suspect, that the results could be changed, with a change of the parameters, like the mean life time of the pheromones, the number of ants per colony and most importantly the regeneration rate of the sources. The explanation, which is shown above would indicate, that the regeneration rate of the sources is low in comparison to the devaluation caused by the ants.

\subsection{SBB Network}
To answer the second question we ran our model on the SBB graph, We chose 2000 time steps, 5 train stations (the largest one in the SBB network) and 30 simulations for each strategy. The regeneration rate of the stations is chosen to match the total number of visitors per day of the stations. The results have a strong similarity to the ones found in \ref{results1}. Again the error margin is very high and no conclusive results supporting any of the strategies can be found. 